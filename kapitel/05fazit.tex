\chapter{Rolle der Bayes'schen Netze in der Robotik}
TODO Quelle: http://robotics.stanford.edu/~koller/NIPStut01/tut6.pdf
Nachdem nun bekannt ist, wie ein Bayes'sches Netz beschaffen ist und für was man es verwenden kann, soll dieses Kapitel den Zusammenhang zum maschinellen Lernen und zur Robotik herstellen.
\section{Maschinelles Lernen und Bayes'sche Netze}
Beim maschinellen Lernen von Bayes'schen Netzen unterscheidet man vier Fälle:
\begin{enumerate}

\item Bekannte Struktur, vollständige Datenbasis
\item Unbekannte Struktur, vollständige Datenbasis
\item Unbekannte Struktur 
\end{enumerate}

\section{?}

Lexmed als bsp

https://web.engr.oregonstate.edu/~tgd/classes/534/slides/part6.pdf leseb