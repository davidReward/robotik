\chapter{Rolle der Bayes'schen Netze in der Robotik}
TODO Quelle: http://robotics.stanford.edu/~koller/NIPStut01/tut6.pdf
Nachdem nun bekannt ist, wie ein Bayes'sches Netz beschaffen ist und für was man es verwenden kann, soll dieses Kapitel den Zusammenhang zum maschinellen Lernen und zur Robotik herstellen.
\section{Maschinelles Lernen und Bayes'sche Netze}
Beim maschinellen Lernen von Bayes'schen Netzen unterscheidet man vier Fälle:
\begin{enumerate}

\item Bekannte Struktur, vollständige Datenbasis
\item Unbekannte Struktur, vollständige Datenbasis
\item Bekannte Struktur, unvollständige Datenbasis 
\item Unbekannte Struktur, unvollständige Datenbasis
\end{enumerate}

Im Hinblick auf den Hintergrund, wie Rechner Bayes'sche Netze lernen können, verweise ich auf das sogenannte \textit{Eager-Lerning}. Mit dessen Hilfe, ist es möglich die oben genannten Szenarien zu erlernen. 

\section{Bayes'sche Netze in der Praxis}
Es existieren viele wissenschaftliche Schriften (papers), die zeigen, wie man Bayes'sche Netze in der Robotik nutzen kann. Im folgenden sind Referenzen zu einigen Schriften zu finden:
\begin{itemize}
\item Objekte mit einer hohen Genauigkeit zu erkennen ist ein Problem in der Robotik bzw. Bildverarbeitung. Diese Schrift schlägt eine konkrete Lösung unter Benutzung eines Bayes'schen Netzes vor: \url{http://www.ri.cmu.edu/pub_files/pub4/schneiderman_henry_2004_2/schneiderman_henry_2004_2.pdf}
\item Für das in Kapitel 3.1 aufgegriffene Problem (Umgang mit falschen Sensordaten) wird in diesen Schrift eine Lösung unter Zuhilfenahme eines Bayes'schen Netzes vorgeschlagen \subitem \url{http://www.eejournal.ktu.lt/index.php/elt/article/viewFile/152/111} 
\subitem \url{https://trac.v2.nl/export/7875/andres/Documentation/Bayesian%20reasoning-networks/dynamic%20bayes%20net%20approach%20to%20multimodal%20sensor%20fusion.ps}
\item 
\end{itemize}



%http://neithan.weebly.com/uploads/5/2/8/0/52807/tfdg_memory.pdf
Semiautomatischer Rollstuhl:
%http://citeseerx.ist.psu.edu/viewdoc/download?doi=10.1.1.85.996&rep=rep1&type=pdf

Balltracking:
%https://courses.cs.washington.edu/courses/cse473/05sp/schedule/15-localization.ppt



