\chapter{Stand der Forschung}
\section{Einführung in die KI}
Nach Schauß und Sabel
TODO
\subsection{Menschliches Handeln}
Ein Ziel der künstlichen Intelligenz kann sein, menschliches Handeln mithilfe von Robotern nachzuahmen. Dies können Aspekte sein, die beispielsweise momentan nur für Menschen möglich sind oder in denen der Mensch Computern überlegen ist.    

\subsection{Menschliches Denken}
Um künstlich das menschliche Denken nachzubilden, muss  man erst einmal verstehen, wie der Mensch überhaupt denkt. Dies ist nicht einfach herauszufinden. Modelle die, die Verhaltensforscher durch Versuche aufstellen,  können hier helfen. Die Modelle können Systeme darstellen. Ein System ist in diesem Kontext gekennzeichnet durch eine Ein- und eine Ausgabe.Gleichen nun Ein- und Ausgabe dem menschlichen Denken, ist das ein positives Indiz für die Eignung des Modells.

Wesentliche Fragen auf diesem Gebiet sind:
\begin{itemize}
\item Bildverarbeitung und Gestenerkennung beim Menschen
\item Sprachverarbeitung beim Menschen
\end{itemize}    

Man verfolgt hier das Ziel, kognitiv adäquate Systeme zu bauen. 
\

\section{Maschinelles Lernen}

\begin{table}
\caption{ Kursübersicht}
\begin{tabular}{|c|c|c|}
\hline
Zeitpunkt & Kursleiter & Titel \\
\hline 
 ...   
\end{tabular}
\end{table}