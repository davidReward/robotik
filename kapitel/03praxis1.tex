\chapter{Bayes'sches Netz}
Dieses Kapitel beschäftigt sich eingangs mit den Grundlagen von Bayes'schen Netzen. Ein Beispiel soll den Sachverhalt greifbar erklären.
Später kann dann der Zusammenhang zwischen den Bayes'schen Netzen, dem maschinellen Lernen und der Robotik gesehn werden.
\section{Fallbeispiel}
In der Vorlesung thematisiert, dass ein Roboter zur Lokalisierung über mehrere Sensorsysteme verfügt. Sei nun \textit{n} die Anzahl der Sensorsysteme und \textit{m} der Wertebereich der möglichen Trefferwahrscheinlichkeiten der Sensorsysteme, dann beträgt der Aufwand der Auswertung des stochastischen Ausdrucks \[ m^n\]. Somit wächst der Aufwand von der Auswertung des Ausdrucks exponentiell. Dieser Aufwand kann in der Praxis glücklicherweise jedoch meist minimiert werden. Oft kann man auf Variablen verzichten, ohne einen Informationsverlust herbeizurufen. Hier kommen Bayes'sche Netze ins Spiel. Sie verwenden das das Wissen über unabhängige Variablen. 
\subsection{Unabhängige Variablen}
Sind Zufallsvariablen paarweise unabhängig, so gilt: \[ P(X_1, ..., X_n) = P(X_1) * P(X_2) * ... * P(X_n)\] 
Betrachtet man unter, dass paarweise Unabhängigkeit gilt folgenden Ausdruck, so kann man 





TODO Quelle: Grundkurs

